\chapter{Using the Auckland Thesis class}
%
% Demonstration text and user guide for aucklandthesis.cls
% by Alistair Kwan, February 2016
%

The \texttt{aucklandthesis} LaTeX class provides a simple, widely applicable infrastructure for typesetting University of Auckland theses. It adheres to the School of Graduate Studies style guide, 2015. It (and this user manual) were written by Dr Alistair Kwan in \textsc{CLeaR}, the Centre for Learning \& Research in Higher Education. Dr Kwan has a mixed background so has designed the class to accommodate science and humanities needs alike. Please do check that, as University regulations progress, the output generated by this class meets your needs. Updated class files may also be issued from time to time.

% ===========================================
\section{Incipit: declaring the class and options.}
% ===========================================
Every thesis written using this class begins as follows:

\begin{verbatim}
   \documentclass[11pt]{aucklandthesis}
\end{verbatim}

The \verb+11pt+ can be changed to \verb+10pt+ or \verb+12pt+ for smaller or larger text, as needed. This parameter specifies the size of the main text; all other text — headings, footnotes, etc. — will be scaled to match. Several further options can also be specified in the brackets:

\smallskip
\noindent
\begin{tabular}{ll}
\verb+examcopy+ & For examination copies; prints an extra paragraph on the title page. \\
\verb+partial+ & For theses that partially fulfil degree requirements; adds the word, \\
	& `partial' to the title page.\\
\verb+doublespace+ & Wider line spacing.\\
\verb+singlespace+ & Narrower line spacing (not acceptable for submission, but may be \\
	& useful for drafts).\\
\verb+oneside+ & Single-sided printing. The default is double-sided.\\
\end{tabular}


% ===========================================
\section{Optional packages}
% ===========================================
Because the class has to cater for such a wide range of disciplinary needs, it in fact does very little. Many students will need to add LaTeX packages appropriate to their work, for example:

\begin{description}
\item {\verb+\usepackage[utf8]{inputenc}+} Allows you to work with files typed directly in Unicode characters. In other words, you can type accented Latin characters (needed in most European languages) instead of encoding them as compounds of accent plus letter. You can still encode as compounds if you prefer: \verb+\v s+, \verb+\' e+, \verb+\~ n+, \verb+\ss+, \verb+\=a+, for example, produce \v s, \'e, \~n, \ss, \=a.

\item {\verb+\usepackage{graphicx}+} allows you to insert graphics files. These could be scanned photographs, charts from Excel or MatLab, diagrams from Illustrator. The preferred formats are png, jpg and pdf. 

\item {\verb+\usepackage{appendix}+} allows you to add appendices with their own numbering sequence.

\item {\verb+\usepackage{booktabs}+} provides good rules for tables.

\item {\verb+\usepackage{microtype}+} improves text justification.

\end{description}

% ===========================================
\section{Changing the typeface}
% ===========================================

The default typeface in LaTeX is well known in mathematical disciplines, but a bit unusual elsewhere. If it does not suit your needs, you can change it by choosing one of the following:

\begin{description}
\item {\verb+\usepackage{mathptmx}+} selects Times New Roman, including for mathematics.

\item {\verb+\usepackage{mathpazo}+} selects Palatino, including for mathematics.

\item {\verb+\usepackage{fourier}+} selects Fourier, another serif typeface with mathematics support.

\item {\verb+\usepackage{gentium}+} selects Gentium, a contemporary crisp serif typeface strong on multilingual capacities, but retains the default mathematics typeface.

\item {\verb+\usepackage{libertine}+} selects Linux Libertine, a conservative, soft-cornered serif typeface without changing the default mathematics typeface.

\end{description}

In general, typefaces are very difficult to change in LaTeX, but a recent variant, called XeLaTeX, makes typeface selection much easier. With XeLaTeX and files saved using the Unicode system (menu options might read `UTF-8' or `UTF-16'), you can use nearly any font file on your computer, and type directly in multiple languages including Chinese, Japanese, Hebrew, Greek, Egyptian, Old English. M\=aori macrons are no problem at all.

XeLaTeX is still in a relatively early stage of development, however, so it has not been selected as the default compiler for the Auckland Thesis class. If you want to use it, you will have to select it yourself.

If you need only a few symbols, for example in a thesis on linguistics, electronics, astrology, music or alchemy, see the Comprehensive LaTeX Symbol List at \url{https://www.ctan.org/tex-archive/info/symbols/comprehensive/}


% ===========================================
\section{Declare your degree intentions}
% ===========================================

\begin{description}
\item {\verb+\title{}+} The thesis title.
\item {\verb+\author{}+} Your name, in its full legal format. 

Many students include postnominal degree abbreviations. There are multiple strict abbreviation protocols — one each from Oxford and Cambridge in widespread international use, another from the Association of Commonwealth Universities, plus Auckland's own system for internally produced documents. Your thesis is a submission for examination, not a University-produced document, so you can choose for yourself whether to follow the University's abbreviation system.  
\item {\verb+\degreesought{}+} Most likely `Doctor of Philosophy', `Master of Arts' or `Master of Science' written out in full, but this class accommodates also other doctors', masters' and bachelors' degrees with honours. It does not accommodate certificates or diplomas. Copy the degree name from the University Calendar.
\item {\verb+\degreediscipline{}+} The title page specifies the discipline, not the department (or institute or school). Copy the discipline name from the Calendar.
\item {\verb+\degreecompletionyear{}+} Enter here the year in which you complete all requirements for admission to the degree. (Your actual graduation year could be later, or you could choose not to graduate at all.)
\end{description}

The title page is composed using the command, \verb+\maketitlepage+.

% ===========================================
\section{Table of contents}
% ===========================================
\begin{description}
\item {The \texttt{tocdepth} counter}

\item {Lists of figures and tables}
\end{description}
% ===========================================
\section{Incorporating thesis content}
% ===========================================
It is easier to manage your thesis if all of your lengthy content elements are saved in separate files, one for each of the abstract, each chapter, each appendix.

\begin{enumerate}
\item Abstract. Begin the file with the command, \verb+\chapter*{Abstract}+ The asterisk prevents the heading from being numbered or entered into the table of contents. If you want it in the table of contents, do it like this:

\noindent
\verb+\chapter*{Abstract}+ \\
\verb+\addcontentsline{toc}{chapter}{Abstract}+

The argument \verb+toc+ there stands for `Table of Contents'. You can also add entries at the section or subsection levels by switching out `\verb+chapter+'.

\item Dedication. The dedication is usually very short, there is a command for entering it directly into your master document rather than as an external file: \verb+\thesisdedication{}+

\item Acknowledgements. Your supervisors are not listed on the title page on the understanding that they did not write your thesis for you, but you may mention them and their contribution here. You might also list librarians, archivists, laboratory technicians, other academics who supported you. Family members, pets and friends are often mentioned in acknowledgements, as are funding agencies.

This section may also be titled `Preface' or `Preface and acknowledgements'. A preface would describe your rationale and perhaps broader motivations for writing this work. The rationale is sometimes personal, sometimes intellectual, and often both. In some disciplines, the preface is the only part of the thesis or book in which you can be completely open about what you are really doing.

\item Glossary. Begin with \verb+\chapter*{Glossary}+.

\item Chapters. Each begins with the \verb+\chapter{}+ command. The headings automatically generate entries in the table of contents. 

	Save each chapter in its own file, and use the \verb+\input{}+ command (or \verb+\include{}+) to incorporate them.

\item Appendices. The appendices are preceded by the command \verb+\appendix+ to get appendix-style headings and numbering when you start each appendix with the \verb+\chapter{}+ command. You must also reset the page counter to 1 at this point.

\item Bibliography.
% ===========================================
\section{University forms}
% ===========================================
Further forms need to be obtained from the University for lodgement and to assert your copyright. This class does not provide those forms; ask at the School of Graduate Studies or look on the Doctoral Skills Program hub.

\end{enumerate}