\documentclass[11pt]{aucklandthesis}
%
% This is a template and demonstration for University of Auckland theses.
%
% Written by Alistair Kwan, February 2016
% 
%
% Options:
%	10pt, 11pt, 12pt: size of main text
% 	examcopy: asserts confidentiality for examination copies
%	partial: thesis partial fulfils degree requirements
%	singlespace, onehalfspace, doublespace: line spacing
%	oneside: format for single-sided printing (double-sided is the default)
%
\usepackage[utf8]{inputenc} % to allow direct entry of accented and other non-English text in Unicode

%\usepackage{graphicx} % for inserting graphics files

%\usepackage{appendix} % for appendices

\usepackage{url} % for formatting web addresses and other URLs
\urlstyle{same} % try also tt, sf if this option doesn't produce clear enough outpout

% Readability options
%
\usepackage{booktabs} % for table rules
\usepackage{microtype} % for improved justification

% Typeface options — choose one if desired
% or choose a different typeface to accommmodate character sets
% as needed for East Asian and other languages.
%
% Consider compiling using the XeLaTeX engine if you have more extreme
% typeface needs, e.g. for multiple languages, or a need for symbols particular
% to a typeface.
%
% See also the LaTeX Symbols List at
% https://www.ctan.org/tex-archive/info/symbols/comprehensive/
%
%\usepackage{mathptmx} % Times New Roman, including mathematics
%\usepackage{mathpazo} % Palatino with mathematics support
%\usepackage{fourier} % a serif typeface with mathematics support
%\usepackage{gentium} % a contemporary serif typeface
%\usepackage{libertine} % a softer-feeling serif typeface

\usepackage{tabularx} % For easier table formatting. This package is needed for 
	% the sample glossary and can be removed if you do not need it.

\usepackage[nottoc]{tocbibind}

% Automated bibliography
%
\usepackage[
	style=authortitle, 
	citestyle=authortitle,
	backend=biber
	]
	{biblatex}
\bibliography{kwan-bibliography.bib}

\begin{document}

% ====================================================
% FRONTMATTER
%
% Arabic pagination, starting with the title page
% which is counted but not numbered
%
% ====================================================

% Specify the title page content
\title{One fish, two fish, red fish, blue fish: \\ a piscine hermeneutics}
\author{Ivana Pa\ss, BSc (Hons)}
\degreesought{Master of Philosophy} 
\degreediscipline{Occult Sciences}
\degreecompletionyear{2015}

% Print the title page
\maketitle

% Abstract, up to 350 words
\chapter*{Abstract}
%
% Demonstration text and user guide for aucklandthesis.cls
% by Alistair Kwan, February 2016
%
My thesis can be summarised in fewer than three hundred and fifty words, raising the question of why I wrote three hundred and fifty pages. 
 % it's in a separate file
I
% Dedication (optional)
\thesisdedication{Dedicated to grandma, and to grammar.}

% Preface and/or acknowledgements
\chapter*{Acknowledgements}
%
% Demonstration text and user guide for aucklandthesis.cls
% by Alistair Kwan, February 2016
%
This thesis would not have been written but for my faithful St Bernard. To him, I owe many a paragraph dragged unerringly through the alpine snows. % it's in a separate file

% Contents, lists of tables and figures
\setcounter{tocdepth}{4} % 2=chapters, 3=sections, 4=subsections 
\cleardoublepage\tableofcontents\markboth{Contents}{}
\cleardoublepage\listoffigures\markboth{Figures}{} % optional
\cleardoublepage\listoftables\markboth{Tables}{} % optional

% Glossary (optional)
\chapter*{Glossary}
\addcontentsline{toc}{chapter}{Glossary}
%
% Demonstration text and user guide for aucklandthesis.cls
% by Alistair Kwan, February 2016
%

% The @{} entry in the tabularx environment makes the table align properly with the left margin.

This glossary file demonstrates how to lay out a table.

\medskip

\noindent
\begin{tabularx}{\textwidth}{@{}lX}
cap. & Abbreviation for `chapter', from Latin `caput'. \\
doctor & The highest degree conferred by British Commonwealth universities. It is the Latin word for `teacher', so named because the original doctorates (established in 1213) were formal licenses to teach
in a university. These were conferred in professional disciplines — medicine, law, theology. The modern research PhD (or DPhil) degree originates in the reorganisation of higher education in 19th-century Germany, spreading to the US in 1861 and to the UK in 1921. \\
em & A typographic unit of relative length, equal to the font size. Typographic spacing is commonly measured in multiples, halves, thirds, quarters and fifths of ems. For the font in use here, a 1-em square — called a `quad' — is this big: \rule{1em}{1em} \\
font & A typeface, typestyle, and inter-baseline spacing. For example, ``Times New Roman Italic 10/12'' is at the 10-point size, with 2 points of interlinear spacing (``leading'') added to achieve 12 points between successive baselines. \\
font size & Specified in points (q.v.), and equal to the height of the metal pieces used for printing. Many typefaces do not fill the whole space available, many typefaces have small x-heights (q.v.), and some even overhang the type edge, so font size does not actually tell you the size of the text. \\
italic & A typeface category: based on the handwriting of Renaissance Italy. \\
point & A typographic unit of absolute length, commonly equal to $\frac{1}{72}$ of an inch. The following lines are 8, 9, 10, 11, 12 pt high: \rule{1pt}{8pt} \rule{1pt}{9pt}  \rule{1pt}{10pt} \rule{1pt}{11pt} \rule{1pt}{12pt} \\
pt & Abbreviation for `point' (q.v.) \\
roman & A typeface category: based (more or less) on the writing and carved lettering of ancient Rome. The archetype is on Trajan's column. \\ 
rule & A typographic line. Its thickness is called \emph{weight} and is typically measured in points. \\
thesis & An argument for or against an hypothesis. Metaphorically, a dissertation for degree examination. \\
x-height & The height of the letter x in a font. \\
\end{tabularx}

% ====================================================
% MAINMATTER
%
% Include external chapter files here using
% the \input{} command
%
% ====================================================
\chapter{Using the Auckland Thesis class}
%
% Demonstration text and user guide for aucklandthesis.cls
% by Alistair Kwan, February 2016
%

The \texttt{aucklandthesis} LaTeX class provides a simple, widely applicable infrastructure for typesetting University of Auckland theses. It adheres to the School of Graduate Studies style guide, 2015. It (and this user manual) were written by Dr Alistair Kwan in \textsc{CLeaR}, the Centre for Learning \& Research in Higher Education. Dr Kwan has a mixed background so has designed the class to accommodate science and humanities needs alike. Please do check that, as University regulations progress, the output generated by this class meets your needs. Updated class files may also be issued from time to time.

% ===========================================
\section{Incipit: declaring the class and options.}
% ===========================================
Every thesis written using this class begins as follows:

\begin{verbatim}
   \documentclass[11pt]{aucklandthesis}
\end{verbatim}

The \verb+11pt+ can be changed to \verb+10pt+ or \verb+12pt+ for smaller or larger text, as needed. This parameter specifies the size of the main text; all other text — headings, footnotes, etc. — will be scaled to match. Several further options can also be specified in the brackets:

\smallskip
\noindent
\begin{tabular}{ll}
\verb+examcopy+ & For examination copies; prints an extra paragraph on the title page. \\
\verb+partial+ & For theses that partially fulfil degree requirements; adds the word, \\
	& `partial' to the title page.\\
\verb+doublespace+ & Wider line spacing.\\
\verb+singlespace+ & Narrower line spacing (not acceptable for submission, but may be \\
	& useful for drafts).\\
\verb+oneside+ & Single-sided printing. The default is double-sided.\\
\end{tabular}


% ===========================================
\section{Optional packages}
% ===========================================
Because the class has to cater for such a wide range of disciplinary needs, it in fact does very little. Many students will need to add LaTeX packages appropriate to their work, for example:

\begin{description}
\item {\verb+\usepackage[utf8]{inputenc}+} Allows you to work with files typed directly in Unicode characters. In other words, you can type accented Latin characters (needed in most European languages) instead of encoding them as compounds of accent plus letter. You can still encode as compounds if you prefer: \verb+\v s+, \verb+\' e+, \verb+\~ n+, \verb+\ss+, \verb+\=a+, for example, produce \v s, \'e, \~n, \ss, \=a.

\item {\verb+\usepackage{graphicx}+} allows you to insert graphics files. These could be scanned photographs, charts from Excel or MatLab, diagrams from Illustrator. The preferred formats are png, jpg and pdf. 

\item {\verb+\usepackage{appendix}+} allows you to add appendices with their own numbering sequence.

\item {\verb+\usepackage{booktabs}+} provides good rules for tables.

\item {\verb+\usepackage{microtype}+} improves text justification.

\end{description}

% ===========================================
\section{Changing the typeface}
% ===========================================

The default typeface in LaTeX is well known in mathematical disciplines, but a bit unusual elsewhere. If it does not suit your needs, you can change it by choosing one of the following:

\begin{description}
\item {\verb+\usepackage{mathptmx}+} selects Times New Roman, including for mathematics.

\item {\verb+\usepackage{mathpazo}+} selects Palatino, including for mathematics.

\item {\verb+\usepackage{fourier}+} selects Fourier, another serif typeface with mathematics support.

\item {\verb+\usepackage{gentium}+} selects Gentium, a contemporary crisp serif typeface strong on multilingual capacities, but retains the default mathematics typeface.

\item {\verb+\usepackage{libertine}+} selects Linux Libertine, a conservative, soft-cornered serif typeface without changing the default mathematics typeface.

\end{description}

In general, typefaces are very difficult to change in LaTeX, but a recent variant, called XeLaTeX, makes typeface selection much easier. With XeLaTeX and files saved using the Unicode system (menu options might read `UTF-8' or `UTF-16'), you can use nearly any font file on your computer, and type directly in multiple languages including Chinese, Japanese, Hebrew, Greek, Egyptian, Old English. M\=aori macrons are no problem at all.

XeLaTeX is still in a relatively early stage of development, however, so it has not been selected as the default compiler for the Auckland Thesis class. If you want to use it, you will have to select it yourself.

If you need only a few symbols, for example in a thesis on linguistics, electronics, astrology, music or alchemy, see the Comprehensive LaTeX Symbol List at \url{https://www.ctan.org/tex-archive/info/symbols/comprehensive/}


% ===========================================
\section{Declare your degree intentions}
% ===========================================

\begin{description}
\item {\verb+\title{}+} The thesis title.
\item {\verb+\author{}+} Your name, in its full legal format. 

Many students include postnominal degree abbreviations. There are multiple strict abbreviation protocols — one each from Oxford and Cambridge in widespread international use, another from the Association of Commonwealth Universities, plus Auckland's own system for internally produced documents. Your thesis is a submission for examination, not a University-produced document, so you can choose for yourself whether to follow the University's abbreviation system.  
\item {\verb+\degreesought{}+} Most likely `Doctor of Philosophy', `Master of Arts' or `Master of Science' written out in full, but this class accommodates also other doctors', masters' and bachelors' degrees with honours. It does not accommodate certificates or diplomas. Copy the degree name from the University Calendar.
\item {\verb+\degreediscipline{}+} The title page specifies the discipline, not the department (or institute or school). Copy the discipline name from the Calendar.
\item {\verb+\degreecompletionyear{}+} Enter here the year in which you complete all requirements for admission to the degree. (Your actual graduation year could be later, or you could choose not to graduate at all.)
\end{description}

The title page is composed using the command, \verb+\maketitlepage+.

% ===========================================
\section{Table of contents}
% ===========================================
\begin{description}
\item {The \texttt{tocdepth} counter}

\item {Lists of figures and tables}
\end{description}
% ===========================================
\section{Incorporating thesis content}
% ===========================================
It is easier to manage your thesis if all of your lengthy content elements are saved in separate files, one for each of the abstract, each chapter, each appendix.

\begin{enumerate}
\item Abstract. Begin the file with the command, \verb+\chapter*{Abstract}+ The asterisk prevents the heading from being numbered or entered into the table of contents. If you want it in the table of contents, do it like this:

\noindent
\verb+\chapter*{Abstract}+ \\
\verb+\addcontentsline{toc}{chapter}{Abstract}+

The argument \verb+toc+ there stands for `Table of Contents'. You can also add entries at the section or subsection levels by switching out `\verb+chapter+'.

\item Dedication. The dedication is usually very short, there is a command for entering it directly into your master document rather than as an external file: \verb+\thesisdedication{}+

\item Acknowledgements. Your supervisors are not listed on the title page on the understanding that they did not write your thesis for you, but you may mention them and their contribution here. You might also list librarians, archivists, laboratory technicians, other academics who supported you. Family members, pets and friends are often mentioned in acknowledgements, as are funding agencies.

This section may also be titled `Preface' or `Preface and acknowledgements'. A preface would describe your rationale and perhaps broader motivations for writing this work. The rationale is sometimes personal, sometimes intellectual, and often both. In some disciplines, the preface is the only part of the thesis or book in which you can be completely open about what you are really doing.

\item Glossary. Begin with \verb+\chapter*{Glossary}+.

\item Chapters. Each begins with the \verb+\chapter{}+ command. The headings automatically generate entries in the table of contents. 

	Save each chapter in its own file, and use the \verb+\input{}+ command (or \verb+\include{}+) to incorporate them.

\item Appendices. The appendices are preceded by the command \verb+\appendix+ to get appendix-style headings and numbering when you start each appendix with the \verb+\chapter{}+ command. You must also reset the page counter to 1 at this point.

\item Bibliography.
% ===========================================
\section{University forms}
% ===========================================
Further forms need to be obtained from the University for lodgement and to assert your copyright. This class does not provide those forms; ask at the School of Graduate Studies or look on the Doctoral Skills Program hub.

\end{enumerate} 
\chapter{Writing your thesis in LaTeX}
%
% Demonstration text and user guide for aucklandthesis.cls
% by Alistair Kwan, February 2016
%
% ===================================================
\section{Common commands}
% ===================================================
\begin{description}
\item {\verb+\chapter{}+} marks a new chapter. LaTeX handles the numbering automatically, as it also does for \verb+\section{}+, \verb+\subsection{}+, \verb+\subsubsection{}+. For University of Auckland theses, avoid dividing any further than sub-subsections.
\item {\verb+\chapter*{}+}, with an asterisk, produces an \emph{un-numbered} heading at the same level as the chapter. It does not appear in the table of contents. You can  similarly add an asterisk to the section, subsection and sub-subsection commands.
\item {\verb+\emph{}+} is the usual command for emphasing text by \emph{italicisation}. There is another way, but this way is better.
\item {\verb+\textbf{}+} makes \textbf{boldface text}.
\item {\verb+\footnote{}+} inserts footnotes. These are automatically numbered.
\item {\verb+-+} inserts a hyphen: - 
\item {\verb+--+} inserts an en dash: --
\item {\verb+---+} inserts an em dash: ---
\end{description}

That is probably enough to get you started. If you read the source code for this chapter, you'll find examples of other commands as well.

% ===================================================
\section{Inserting figures and tables}
% ===================================================
\label{sec:figuresandtables}
Figures and tables are inserted using the \verb+figure+ and \verb+table+ \emph{environments}. These differ from the formatting commands listed above: the special thing about figures and tables is that they float around, so the LaTeX compiler finds the best place for them. 

\begin{figure}[tp]
\begin{verbatim}
     \begin{figure}[tp]
          \includegraphics[width=5cm]{horse}
          \caption[A horse.]{A picture of a horse, and a 
          scholarly observation about how the tail seems 
          always to be at the end opposite the head.}
          \label{fig:horsepicture}
     \end{figure}
\end{verbatim}\caption{The figure environment.}
\label{fig:figure}
\end{figure}

An example of the figure environment is shown in figure \ref{fig:figure}. Line by line, it works like this:
\begin{description}
	\item{\verb+\begin{figure}[tp]+} opens the environment for this figure. The \verb+t+ in \verb+[tp]+ says to put the figure at the top of a page if possible. If not possible, the \verb+[p]+ says put it on a page of its own.
	\item{\verb+\includegraphics[width=5cm]{horse}+} brings in a picture from a separate file. It's probably called horse.png or horse.jpg (the LaTeX compiler will work it out). It will appear 5cm wide. You could alternatively set the picture's height. Or you could set both, which can be handy if you want to adjust its aspect ratio.
	\item {The \verb+\caption+ command} has two parts. The bit in curly braces gets printed under the figure. This is too long for the table of figures, so the optional part in square brackets specifies what to use there. 
	\item {\verb+\label{fig:horsepicture}+} gives you an anchor for cross-references. Wherever you need the figure number in the document, use \verb+\ref{fig:horsepicture}+. The numbering automatically updates as needed. Many of us start figure labels with \texttt{fig:} to help us mentally keep track of the labels — and also \texttt{tab:}, \texttt{eq:}, \texttt{ch:}, \texttt{sec:} for labelling tables, equations, chapters, sections.  You, however, can call them whatever you like. The LaTeX compiler doesn't care very much.
	
	This is section \ref{sec:figuresandtables}, referred to as \verb+\ref{sec:figuresandtables}+.
	
	\texttt{label} commands come immediately after whatever \emph{numbered} thing it is that they're labelling. In a figure or table, that's the caption — not the picture or the table.
	\item{\verb+\end{figure}+} completes the figure environment.
\end{description}

Notice that the figure caption comes below the picture. If this were a table, the caption would go at the top, as the first command inside the table environment. So table environment looks like the example in figure \ref{fig:table}. The table environment does not produce the table; it only ensures that the caption gets a table numbering and that the contents entry goes to the list of tables. The table itself is constructed using a \textbf{tabular} environment.

\begin{figure}[tp]
\protect\begin{verbatim}
     \begin{table}[tp]
          \caption{Food for the horse.}
          \label{tab:horsefoods}
          \begin{center}
             \begin{tabular}{llc}
             \toprule
             Food & Colour & Quantity \\
             \midrule
             grass & green & 7 kg \\
             sugar & white & 2 cubes \\
             carrot & orange & 14 \\
             \bottomrule
             \end{tabular}
          \end{center}
     \end{table}
\end{verbatim}
\caption{The table environment, with a tabular environment inside.}
\label{fig:table}
\end{figure}

% ===================================================
\section{Equations}
% ===================================================

LaTeX has a whole sub-language for describing mathematical notation. This code:

\begin{verbatim}
\begin{equation}
     -\frac{\hbar^2}{2m} \frac{\partial \psi}{\partial x} + V \psi 
       = i \hbar \frac{\partial \psi}{\partial t}
     \label{eq:schrodinger}
\end{equation}
\end{verbatim}
produces this:

\begin{equation}
	-\frac{\hbar^2}{2m} \frac{\partial \psi}{\partial x} + V \psi = i \hbar \frac{\partial \psi}{\partial t}
	\label{eq:schrodinger}
\end{equation}

The equation numbers at the right margin are automatic.

Function names are given by commands, so that readers perceive them as functions rather than as a run of variables, viz. $cos x$ and $sin a$ vs $\cos x$ and $\sin a$.

\begin{equation}
	\int^a_b \cos x\; dx = \sin a - \sin b + c
\end{equation}

Notice also that there's a big space between $\cos x$ and the $dx$ making the integral easier to read at a glance than $\int\cos x dx$. You specify that space using the command, \verb|\;|. You can also make narrower spaces with \verb|\,| or \verb|\.| and a wider space with \verb|\:|.

If you need mathematical notation within your paragraphs instead of as stand-alone paragraphs, use dollar signs to invoke `math mode' like this:  \verb|$F = m \times a$|.

The dollar sign always does that in LaTeX. If you need to write about all your \$\$\$, put a backslash before: \verb|\$|.


% ===================================================
\section{Verse and quotations}
% ===================================================

LaTeX provides \texttt{verse}, \texttt{quote} and \texttt{quotation} environments. All three of these indent at both the left and the right.

\begin{verbatim}
\begin{verse}
L'autre jour, au fond d'un vallon \\
un serpent piqua Jean Fréron.

Que pensez-vous qu'il arriva? \\
Ce fut il serpent qui creva. \\

\hfill --- Voltaire.
\end{verse}
\end{verbatim}

The double slash marks the end of a line, just as it does in the \texttt{tabular} environment. A blank line separates the stanzas.

The \verb+\hfill+ command fills out the line horizontally, pushing the poet's name to the far right.  The result is as follows:

\begin{verse}
L'autre jour, au fond d'un vallon \\
un serpent piqua Jean Fréron.

Que pensez-vous qu'il arriva? \\
Ce fut il serpent qui creva. \\

\hfill --- Voltaire.
\end{verse}

Try the \texttt{quote} and \texttt{quotation} environments yourself.

% ===================================================
\section{Custom characters}
% ===================================================
Occasionally scholarship requires special characters, such as symbols for units of currency or weight, or astrological or alchemical signs. Many of these are provided by LaTeX packages (look on-line for the \emph{LaTeX Symbol List}). Most present-day mathematical symbols are built-in.


% ===================================================
\section{Typographic trivia}
% ===================================================
\subsection{Standard conventions}
`Serious' books follow most or all of the following conventions:
\begin{itemize}
\item Odd pages are always on the right, even pages on the left.
\item Chapters and the table of contents always start on odd pages.
\item When the first lines of paragraphs are indented, an exception is made for the first paragraph after each heading.
\item When paragraphs are marked by indenting a line, they are not spaced out. (LaTeX often breaks this rule.)
\item No underlining.
\item Tables have only horizontal rules, not vertical rules.
\item Captions are above tables, and below figures.
\end{itemize}

\subsection{Why are books paginated in roman numerals, then in arabic numerals?}
In the old days of printing by hand, you could not write the table of contents until you knew which pages everything would be on. That had to wait until the text was typeset by arranging little metal letters, page by page. So this — the mainmatter — was done first. Afterwards, the frontmatter was typeset — a title page, the table of contents, and prefaces and dedications. The prefaces and dedications were also often left until the end in case political and financial imperatives necessitated a last-minute change. All of this required that the frontmatter by paginated separately. The French went a step further, putting the table of contents at the end, often on the back cover.

The endmatter often has its own page numbering as well. Often the endmatter is done completely separately, for example as an advertising supplement.

Now that computers re-typeset and adjust the page numbers on the fly, it is no longer necessary to paginate the three parts separately. But many printers still do, because many readers still subconsciously anticipate the long-established norms.

\subsection{Why two title pages?}
Books usually have both a full title page and a short title page. The full page used to be what customers saw when browsing a book stall, back when books were sold without covers. Old title pages usually have a lot of content because they serve the same advertising pitch as today's dust jackets. The short title page is believed to date from back when books were stored flat on the shelf, with their spines facing the wall. The short title page was there to be torn out and folded over the fore-edge, providing a way to tell which book was which.

\subsection{Readability}
Typography might look like it's about creativity and personal expression but, even more, it's about readability. One of the primary criteria in book typography is that, if your readers are paying attention to your ingenious font choice, they're not paying attention to your thesis. So, as you develop your LaTeX skills, be discreet. Use typography to help get your content across.
 
%
% Demonstration text and user guide for aucklandthesis.cls
% by Alistair Kwan, February 2016
%

\chapter{Bibliographies with BibLaTeX}

Compiling a bibliography in LaTeX requires some investment, just as it would if you wrote your thesis using other tools. The demonstration here shows how to use a supplementary application called \emph{BibLaTeX}.  BibLaTeX offers considerable flexibility in automating the citation style, but here you will see only a beginning: BibLaTeX does much more than this.  For example, BibLaTeX allows  historians to separate published primary, unpublished primary, and secondary sources into three separate listings.

If your thesis involves only a short list of citations, and you don't need to cite very heavily (e.g. in a thesis that predominantly analyses your data), you could use the simpler features of BibTeX. 

For further details about LaTeX's internal bibliography system, BibTeX and BibLaTeX, see \url{https://en.wikibooks.org/wiki/LaTeX/Bibliography_Management}


% =================================================
\section{Infrastructure}
% =================================================

BibLaTeX needs a few things in your master document. 

\subsection{Preamble}

In the preamble, add this:

\begin{enumerate}
\item \verb+\usepackage[+

	\quad\quad \verb+style=authoryear,+ 

	\quad\quad \verb+citestyle=authortitle,+
	
	\quad\quad \verb+backend=biber+

	\quad \verb+]{biblatex}+

\item \verb+\addbibresource{kwan-bibliography}+
\end{enumerate}

The first command loads the BibLaTeX package. It sets the bibliography style and the citation style. The styles here set the bibliography to be sorted by author, then year, and citations by author and title (see section \ref{sec:citationstyle} below.) The `back end', i.e. the engine that does database handling, is set to Biber, which handles multilingual work much better than BibTeX can. You probably don't need to know much more about Biber than that.

The second command nominates your bibliography file (ending in \emph{.bib}). You can name multiple files here. This particular file is one of several that went into Alistair Kwan's doctoral dissertation. It contains a mixture of entries so you can see what a realistic \emph{.bib} file might look like.

\subsection{Endmatter}

At the very end of your master file, where you want the bibliography to appear, add this:

\begin{itemize}
\item \verb+\printbibliography+
\end{itemize}


% ============================================
\section{Citing other people's work}
% =================================================

The main citation command is \verb+\cite{}+. It comes in a few variants:

\subsection{Citing a work, a chapter or a page range}
\newlength{\tablecolwidth}
\setlength{\tablecolwidth}{0.5\columnwidth}
\addtolength{\tablecolwidth}{-2\tabcolsep}
\begin{tabular}{p{\tablecolwidth}p{\tablecolwidth}}
\toprule
Citing like this... & Does this: \\
\midrule
\verb+\cite{Bringhurst1999}+ 
	& \cite{Bringhurst1999} \\
\verb+\cite[12, 14--16]{Schmutz2006}+
	& \cite[12, 14--16]{Schmutz2006} \\
\verb+\cite[chap. 3]{Lassels1698}+
	& \cite[chap. 3]{Lassels1698} \\
\verb+\cite[sec.~6]{Wren1669}+ 
	& \cite[sec.~6]{Wren1669} \\
\verb+\cite{+\texttt{Bringhurst1999, Nielsen1968, Chambers1728}\verb+}+ 
	& \cite{Bringhurst1999, Nielsen1968, Chambers1728} \\
\verb+\parencite[72]{Bringhurst1999}+
	& \parencite[72]{Bringhurst1999} \\
\verb+\Textcite[72]{Bringhurst1999}+ 
	& \Textcite[72]{Bringhurst1999} \\
\bottomrule
\end{tabular}


\subsection{Citing with style}
\label{sec:citationstyle}

BibLaTeX comes with a lot of in-built styles including the famously difficult style of the Chicago University Press. There are more that you can download, and you can make your own.

Why so many styles?  In some disciplines, you discuss other researchers' work by referring to the researchers by name, and their work by date.  In-line citations by name and year serve well in a he-said, she-said discourse.  This is especially helpful when you are referring to very recent work in a fast-paced disciplinary conversation, playing authors off against each other or setting their work up for your coming coup de gr\^ace.

When publication years don't matter so much, e.g. in disciplines in which timeless ideas stick around for a years, decades or centuries of reflection and critique, the author's name often matters but citing the year is silly.  You might also feel a need to cite titles. That makes in-line citation very unwieldy, so often a shortened title is used, and the citation goes into footnotes or endnotes rather than in the body text.

Citing in notes also gives plenty of space in which to summarise your position on a work. Perhaps you want to mention it because the title claims that it tackles your topic, for example, and excuse yourself from attending to it any further because its message, presuming that has one, has proven too rarefied to grasp.

Footnotes or endnotes? Footnotes are usually easier for the reader to consult because they're right there on the same page when you need them.  LaTeX does a very good job of arranging them.  But if you are writing in a way that doesn't need citations to be so immediately present, endnotes allow you to get them completely out of the way.

There are three main reasons for choosing your style: what you want your citations to do, what your readers are familiar with, and because your supervisor told you so.

Table \ref{tab:biblatex-styles} shows some of the in-built style options.

\begin{table}[htp]
\caption{BibLaTeX styles.}
\label{tab:biblatex-styles}
\centering
\begin{tabular}{lp{10cm}}
\toprule
\texttt{numeric} & Numbers the entries. \\
\texttt{alphabetic} & Creates an alphanumeric reference for each entry. \\
\texttt{authoryear} 
	& Author and year. \\
\texttt{authoryear-ibid}
	& Author and year, and the erudite \emph{ibidem}. \\
\texttt{authortitle}
	& Author and title. \\
\texttt{verbose} 
	& A full citation at first, and a short form afterwards. \\
\texttt{chem-acs}
	& American Chemical Society. \\
\texttt{nature}
	& \emph{Nature}.\\	
\texttt{science}
	& \emph{Science}.\\	
\texttt{phys}
	& American Institute of Physics\\	
\texttt{chicago-authordate}
	& Chicago. \\	
\texttt{ieee}
	& Institute of Electrical and Electronics Engineers.\\	
\texttt{mla}
	& Modern Languages Association. Now we're talking.\\
\bottomrule
\end{tabular}
\end{table}

\subsection{Listing works not cited}
Sometimes you want a work to appear in the bibliography even though it hasn't been cited.  The command for this is \verb+\nocite{}+.

This document includes the command, 

\quad\quad \verb+\nocite{Imhof1996, Zeiller1655, Delambre1806}+ 

\noindent to get those three works in from the bibliography file, in addition to those cited above.  Jump ahead to the end to see whether it worked.

If you want every listed work included, say \verb+\nocite{*}+.  

\nocite{Imhof1996, Zeiller1655, Delambre1806}

% =================================================
\section{Where do the references go? The \emph.{bib} file}
% =================================================

All of your citations are detailed in a separate \emph{.bib} file. You can get the entries from Google Scholar, Zotero or Refworks.  You can also enter them manually by yourself.

The \emph{.bib} file used for this demonstration is called \texttt{kwan-bibliography.bib}. It's part of Alistair Kwan's dissertation, and contains a mixture of journal articles and books so you can see how bibliographic entries work.




% =================================================
\section{Compiling}
% =================================================
Compiling with a bibliography requires four steps.
\begin{enumerate}
\item Compile with LaTeX.
\item Run Biber to get the citation entries from the \emph{.bib} file.\footnote{Running Biber may require an adjustment to your editor configuration. Some editors are set up to run BibTeX, but BibLaTeX works better with Biber, especially if you are working in multiple languages.  If the compiler complains that it cannot find any citation entries, it is often because you are running BibTeX rather than Biber.}
\item Compile with LaTeX again to form the citations.
\item Compile with LaTeX a third time to finalise the citation references.
\end{enumerate}
 

% ====================================================
% ENDMATTER
%
% Appendices and bibliography 
% Pagination arabic, re-starts at 1
%
% ====================================================
\cleardoublepage % start afresh on a new page
\setcounter{page}{1} % re-sets the page counter
\appendix % appendices begin now
\chapter{Dummy text}
%
% Demonstration text and user guide for aucklandthesis.cls
% by Alistair Kwan, February 2016
%

This appendix contains a standard example of dummy text, which designers use as a place-holder, allowing you to see how the page will be laid out without distractions from reading the actual content (or needing to wait for the author to provide it). Versions of this particular text are widely said to have been used since the sixteenth century.

Lorem ipsum dolor sit amet, laudem mentitum omittantur nam ex. Usu modo sumo an, ea atqui soluta sed. Ne nemore aliquip minimum mei, est ad molestie appareat, vim doming eripuit ex. Diceret partiendo no quo, sea te augue labitur, mei cu feugiat mediocritatem. Paulo sonet deleniti vix no, eam tation rationibus ne. Iusto docendi reprehendunt sea ne, percipit quaestio ut vis, mei ut meis exerci. Commodo erroribus omittantur vis ei, quod cibo legere in nec.

Pri etiam soluta interesset te, dicant repudiandae mei ad. Vim at essent pertinacia. Adhuc tritani deseruisse ei nec. Eum impedit graecis et. Ei his feugiat molestie. Mel et eruditi consequat.

Nec esse veri minim id, mel ad voluptua facilisi, id habeo timeam ius. Per placerat definitionem ea, nam zril volumus invenire ut. Munere vulputate disputando ei mei, mei in lorem offendit. Per erant reprehendunt te, vis singulis pericula an. No nam stet zril, eam iusto vidisse maluisset te, eam cu porro legimus concludaturque. No magna dicat inermis vis.

Sit at equidem tractatos, mucius civibus pri cu. Vel te equidem principes. Cum te primis vidisse eripuit, essent conceptam eum an, labitur sadipscing eam et. Causae incorrupte pro ut, vix prompta denique erroribus at. Ad pro inani aeterno, suas corpora an sed, sumo epicuri sit ea.

Modus lobortis ne quo, id dicit gloriatur nec. Ex sed soluta vocibus, an dignissim scribentur eum, quem epicurei placerat pri ea. Ex sed albucius nominati, eius idque mei ne. Has vocent integre placerat ei, ei solum feugiat usu. 
\chapter{Typeface specimens}

\newcommand{\fontspecimen}{\noindent abcdefghijklmnopqrstuvwxyz 0123456789 fi ff fl ffl st\\ ABCDEFGHIJKLMNOPQRSTUVWXYZ ÅÉÔ \\ Victor jagt zwölf Boxkämpfer quer über den großen Sylter Deich. Forsaking monastic tradition, twelve jovial friars gave up their vocation for a questionable existence on the flying trapeze. Quizdeltagerne spiste jordbær med fløde, mens cirkusklovnen Walther spillede på xylofon. Monsieur Jack, vous dactylographiez bien mieux que votre ami Wolf! \bigskip}

\fontsize{8pt}{8pt}
\selectfont
\fontspecimen

\fontsize{10pt}{10pt}
\selectfont
\fontspecimen

\fontsize{14pt}{14pt}
\selectfont
\fontspecimen

\noindent
\emph{\fontspecimen}


\printbibliography

\end{document}