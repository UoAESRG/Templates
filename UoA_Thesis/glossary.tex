\chapter*{Glossary}
\addcontentsline{toc}{chapter}{Glossary}
%
% Demonstration text and user guide for aucklandthesis.cls
% by Alistair Kwan, February 2016
%

% The @{} entry in the tabularx environment makes the table align properly with the left margin.

This glossary file demonstrates how to lay out a table.

\medskip

\noindent
\begin{tabularx}{\textwidth}{@{}lX}
cap. & Abbreviation for `chapter', from Latin `caput'. \\
doctor & The highest degree conferred by British Commonwealth universities. It is the Latin word for `teacher', so named because the original doctorates (established in 1213) were formal licenses to teach
in a university. These were conferred in professional disciplines — medicine, law, theology. The modern research PhD (or DPhil) degree originates in the reorganisation of higher education in 19th-century Germany, spreading to the US in 1861 and to the UK in 1921. \\
em & A typographic unit of relative length, equal to the font size. Typographic spacing is commonly measured in multiples, halves, thirds, quarters and fifths of ems. For the font in use here, a 1-em square — called a `quad' — is this big: \rule{1em}{1em} \\
font & A typeface, typestyle, and inter-baseline spacing. For example, ``Times New Roman Italic 10/12'' is at the 10-point size, with 2 points of interlinear spacing (``leading'') added to achieve 12 points between successive baselines. \\
font size & Specified in points (q.v.), and equal to the height of the metal pieces used for printing. Many typefaces do not fill the whole space available, many typefaces have small x-heights (q.v.), and some even overhang the type edge, so font size does not actually tell you the size of the text. \\
italic & A typeface category: based on the handwriting of Renaissance Italy. \\
point & A typographic unit of absolute length, commonly equal to $\frac{1}{72}$ of an inch. The following lines are 8, 9, 10, 11, 12 pt high: \rule{1pt}{8pt} \rule{1pt}{9pt}  \rule{1pt}{10pt} \rule{1pt}{11pt} \rule{1pt}{12pt} \\
pt & Abbreviation for `point' (q.v.) \\
roman & A typeface category: based (more or less) on the writing and carved lettering of ancient Rome. The archetype is on Trajan's column. \\ 
rule & A typographic line. Its thickness is called \emph{weight} and is typically measured in points. \\
thesis & An argument for or against an hypothesis. Metaphorically, a dissertation for degree examination. \\
x-height & The height of the letter x in a font. \\
\end{tabularx}